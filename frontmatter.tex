\begin{frontmatter}

%% Title, authors and addresses

%% use the tnoteref command within \title for footnotes;
%% use the tnotetext command for theassociated footnote;
%% use the fnref command within \author or \address for footnotes;
%% use the fntext command for theassociated footnote;
%% use the corref command within \author for corresponding author footnotes;
%% use the cortext command for theassociated footnote;
%% use the ead command for the email address,
%% and the form \ead[url] for the home page:
%% \title{Title\tnoteref{label1}}
%% \tnotetext[label1]{}
%% \author{Name\corref{cor1}\fnref{label2}}
%% \ead{email address}
%% \ead[url]{home page}
%% \fntext[label2]{}
%% \cortext[cor1]{}
%% \affiliation{organization={},
%%             addressline={},
%%             city={},
%%             postcode={},
%%             state={},
%%             country={}}
%% \fntext[label3]{}

\title{Review of shock wave lithotripsy simulations}

%% use optional labels to link authors explicitly to addresses:
%% \author[label1,label2]{}
%% \affiliation[label1]{organization={},
%%             addressline={},
%%             city={},
%%             postcode={},
%%             state={},
%%             country={}}
%%
%% \affiliation[label2]{organization={},
%%             addressline={},
%%             city={},
%%             postcode={},
%%             state={},
%%             country={}}

\author[inst1]{Yangyuanchen Liu}

	\affiliation[inst1]{organization={Department of Mechanical Engineering and Materials Science},%Department and Organization
		    addressline={Duke University},
	    	city={Durham},
		    postcode={27708},
		    state={NC},
		    country={USA}}

% \author[inst2]{Author Two}
% \author[inst1,inst2]{Author Three}

% \affiliation[inst2]{organization={Department Two},%Department and Organization
%             addressline={Address Two}, 
%             city={City Two},
%             postcode={22222}, 
%             state={State Two},
%             country={Country Two}}

\begin{abstract}
This review is devoted to the study of numerical modeling and simulation of shock wave lithotripsy. The numerical modeling of the shock wave lithotripsy has become more popular due to the increasing number of applications, such as the treatment of kidney stones. The process of shock wave lithotripsy includes acoustic wave propagation, bubble collapse dynamics and elasticity (solid equations). Besides the traditional finite element method which is suitable for heat conduction, different numerical techniques are applied to describe acoustic and solid equations, liquid-solid interface and damage evolution (crack nucleation and propagation). In this review, rationale and numerical methods of shock wave lithotripsy have been introduced, discussed and compared.
\end{abstract}

%%Graphical abstract
% \begin{graphicalabstract}
% \includegraphics{grabs}
% \end{graphicalabstract}

%%Research highlights
% \begin{highlights}
% \item Research highlight 1
% \item Research highlight 2
% \end{highlights}

\begin{keyword}
%% keywords here, in the form: keyword \sep keyword
kidney stone \sep shock wave lithotripsy \sep phase field \sep acoustic-structure interaction \sep damage evolution \sep crack nucleation 
%% PACS codes here, in the form: \PACS code \sep code
% \PACS 0000 \sep 1111
%% MSC codes here, in the form: \MSC code \sep code
%% or \MSC[2008] code \sep code (2000 is the default)
% \MSC 0000 \sep 1111
\end{keyword}

\end{frontmatter}